\documentclass[twocolumn]{article}
\usepackage[T1]{fontenc}
\usepackage{cmbright}
\usepackage{xspace}
\usepackage{hyperref}

\newcommand{\org}{NetVersa LLC\xspace}
\newcommand{\app}{Bitcoin Price\xspace}
\newcommand{\platform}{Android\xspace}
\newcommand{\onlyexch}{Mt.~Gox\xspace}

\title{\org \\ \app App for \platform}
\author{Michael Shick \\ \app developer}

\begin{document}

    \maketitle

    \section*{Overview}

    \app for \platform is a clean and functional bitcoin price lookup app based
    on a powerful, flexible, and extensible engine.

    A quick glance is provided to users upon opening the app in the form of
    price statistics and a candlestick chart of recent activity.  An optional
    background service keeps an up-to-date price in the notification bar.
    Homescreen and lockscreen price widgets are included.  Planned features not
    yet implemented include a service to watch for significant price changes
    and candlestick chart widgets.

    Underneath the user interface is a simple but powerful platform for
    fetching bitcoin price data from various bitcoin exchanges.  Fetches are
    automatically performed in the background, and simultaneous requests are
    handled gracefully without duplicating work and putting unnecessary load on
    exchange APIs.  Additionally, when configured to, \app can receive requests
    from and provide data to entirely separate \platform apps through the same
    interface.

    \section*{Features}

    The primary activity of \app displays the most relevant price information
    collected from the default exchange (currently \onlyexch is supported).
    Last price, 24-hour high price, 24-hour low price, and 24-hour volume are
    displayed.  Beneath the price information, a candlestick chart for the last
    24 hours of price data is displayed to round out a quick picture of the
    state of the market.

    The ongoing price notification service requests periodic price updates and
    displays the last trade price in the notification area.  Users can then
    glance quickly at the notification area from within other apps to check on
    the price.  Price data can also be placed on either the home or lock
    screens through simple price widgets.  Placing a widget schedules periodic
    price updates.

    Although at present only \onlyexch is supported, work to add additional
    exchanges is minimized by using swappable exchange modules.  Multiple
    exchanges can be supported simultaneously.

    Whenever a price update is requested by any part of the app, updates happen
    everywhere else (even outside the app if enabled) automatically.  Even if
    the ongoing price notification may be set to only update every hour, if a
    user opens the app to look at the current price, the ongoing notification
    will be updated simultaneously using the same data.

    Where \app really shines over existing price apps is in the
    comprehensiveness of its feature set and the extended potential provided by
    its advanced engine.  Existing bitcoin apps for \platform focus on a single
    feature, like ongoing price notifications in
    \href{https://play.google.com/store/apps/details?id=br.eti.fml.satoshi}{Bitcoin
    Paranoid}, price widgets in
    \href{https://play.google.com/store/apps/details?id=st.brothas.mtgoxwidget}{Bitcoin
    Widget}, or price change notifications in
    \href{https://play.google.com/store/apps/details?id=com.mastah.bitcoinalert}{BitCoin
    Price Alert}.  \app's planned feature set includes all of these features
    and others, most of which are wholly or partly implemented currently.
    Additionally most, if not all, bitcoin \platform apps have UI and graphic
    design as an afterthought, while \app provides a clean and attractive
    interface.  Being usable and pleasing to the eye from the perspective of
    non-technical consumers is key to widespread acceptance and adoption of an
    app.

    \section*{Technical Description}

    \subsection*{Noteworthy Dependencies}

    The modular exchange interface is provided by the
    \href{https://github.com/timmolter/XChange}{XChange} library.  By
    Leveraging an open-source exchange interface library, any time support is
    added to XChange for a new exchange, only a small amount of code and a
    modest package size increase are required to support that exchange in \app.
    The candlestick charts are rendered by
    \href{https://code.google.com/p/stock-chart/}{stock-chart}.

    \subsection*{Fetch Engine}

    While the implementation of individual features is straightforward, the
    underlying engine that powers \app represents a significant leap forward
    for bitcoin apps.

    Data is sent to and from the engine via \platform broadcasts bearing the
    \texttt{ACTION\_FETCH\_REQUEST} and \texttt{ACTION\_FETCH\_RESPONSE}
    intents respectively.  While communicating via intent introduces additional
    requirements like having data types be parcelable, leveraging the enormous
    flexibility of the intent system provides benefits that far outweigh the
    burdens.

    The data payload of the intent describes the kind of data to be fetched and
    from which exchange (e.g.\ \texttt{data://mtgox/market/btc/usd/}).

    When the engine receives a request to fetch data, it first checks to see if
    a fetch of that URI is already underway, and simply ignores the request if
    so.  This solves a number of issues, including duplicate requests when a
    user rotates their phone and an activity is restarted to handle the change.
    Because fetch results are broadcast, the initiator of the ignored request
    will get data just the same when the first request is fulfilled.

    When a fetch is completed and the results are broadcast, any currently
    listening component will receive and act on that data.  Listening
    components can be inside or outside the app itself; a security directive in
    the package manifest can restrict access if necessary.  This organization
    creates a synergy among components interested in the same data: when one is
    updated, all are updated.

    Background services generally schedule periodic fetches when enabled, and
    the fetch scheduler will honor the most frequent schedule period.  This has
    the effect of improving the service of all other background services as
    they are fulfilled along with the most frequent in a single broadcast.

    \section*{Future Work}

    Much remains that could be added to the framework that supports \app.  Most
    of the groundwork has been laid, so work from this point on is inherently
    more fruitful than previous work.
    Farther down the line third party apps, possibly written by developers
    unrelated to \org, could be written to take advantage of \app.  In a
    scenario where third party apps use \app to get their bitcoin data,
    duplication of labor and therefore introduction of bugs could be
    significantly curtailed.

    \subsection*{Within \app}

    \begin{description}
        \item[price change alerts]
            Work has already begun on price change alerts; upon completion,
            users could set thresholds for price changes they deem significant
            and be notified as soon as they happen.
        \item[price history reports]
            By creating a recurring service that computes statistics, a simple
            activity could be added to provide more in-depth statistics to aid
            in purchasing decisions and market technical analysis.
        \item[performance]
            Although \app runs well on even modestly-powered devices, there
            exist many opportunities to improve speed and responsiveness.  In
            particular caching of data values could improve user experience
            across the entire engine.
        \item[exchange support]
            UI fixtures and additional XChange libraries are required to expand beyond \onlyexch.
        \item[currency pair support]
            UI fixtures are required to support currency pairs beyond BTC/USD.
    \end{description}

    \subsection*{In other apps}

    As mentioned previously, apps outside of \app can leverage the fetching
    capabilities of the \app engine.  If desired, services can be extended only
    to \org apps, or to any interested apps as the cornerstone of a common
    bitcoin price system for \platform.  A common price system could both
    improve the bitcoin app ecosystem and bring attention and prestige to \org.

    The reverse relationship is also possible through the intent broadcast
    system: third party apps could extend \app by providing a compatible fetch
    request interface supporting exchanges not handled by \app.  By committing
    to a clean external interface, new exchanges could come online day one with
    plug-in packages that extend \app and allow users to immediately start
    using the new exchange with their preferred user interface and features.

\end{document}
